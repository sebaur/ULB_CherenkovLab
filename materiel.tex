\section{Le matériel expérimental}

\subsection{Scintillateurs}
\label{subsec:scint}

Un scintillateur est un matériau dans lequel se produit un phénomène de scintillation suite aux interactions entre la radiation ionisante et la matière. Lorsqu'une particule traverse le scintillateur, celui-ci absorbe une petite partie de son énergie et la ré-émet ensuite sous forme de lumière. Les scintillateurs présents dans vos dispositifs sont couplés à des photomultiplicateurs afin d'obtenir un détecteur à scintillation.

\begin{figure}[h]
    \centering
	\includegraphics[width=\textwidth]{figures/Scintillateur.png}
    \caption{Schema d'un détecteur à scintillation.}
    \label{fig:Scintillateur} 
\end{figure}

\FloatBarrier

\subsection{Photomultiplicateurs}
\label{subsec:PMT}
Les photmultiplicateurs (PMs) sont des détecteurs basés sur l'effet photoélectrique. Un schema est donné dans figure ~\ref{fig:PMT_readout}. Lorsqu'un photon arrive à la photocathode, il lui arrache un électron par effet photoélectrique. Le faible courant ainsi produit est ensuite amplifié par une série de dynodes afin d'obtenir un gain important. Le gain du photomultiplicateur est défini comme :
\begin{equation}
G = \frac{Q_{1pe}}{q_{e}} 
\end{equation}
où $Q_{1pe}$ est la charge collectée en sortie du photomultiplicateur pour le premier photo-électron (pe) et $q_{e}$ représente la charge d'un électron. La résolution du photomultiplicateur est donnée par :
\begin{equation}
\sigma_{G} =  \frac{\sigma_{1pe}}{Q_{1pe}} * 100
\end{equation}
où $\sigma_{1pe}$ est l'écart-type du pic du premier pe.

\begin{figure}
    \centering
	\includegraphics[width=\textwidth]{figures/PMT_readout.png}
    \caption{Schema du fonctionnement d'un photomultiplicateur(PMT).}
    \label{fig:PMT_readout} 
\end{figure}

Nous allons à présent étudier les différentes composantes d'un spectre en charge typique de la réponse d'un photomultiplicateur, illustré dans la figure~\ref{fig:spectre}.

\textbf{Piédestal :} Il s'agit d'évènements sans charge qui prennent la forme d'un pic en zéro. Afin de se débarrasser de cet effet, il vous faudra régler au mieux votre seuil (voir section~\ref{sec:materiel}, le discriminateur).

\textbf{Dark current :} Il s'agit de bruit associé au PM, il survient lorsqu'un électron est arraché à une dynode sans qu'un photon incident n'arrive à la photocathode. Il nous donne une exponentielle décroissante. Cet effet est exacerbé lorsque la tension aux bornes du photomultiplicateur est élevée.

\textbf{Pics des photo-électrons :} Il s'agit de la réponse en charge du PM pour différents nombres (1,2,3,...) de photo-électrons.

Le largeur de la gaussienne du premier photo-électron (1pe) va nous donner la résolution en charge du PM. La relation entre la hauteur des gaussiennes nous est donnée par la distribution de Poisson.

\begin{figure}
    \center{\includegraphics[width=0.7\textwidth]
    {figures/SpectreEnCharge.png}}
    \caption{\label{fig:spectre} Spectre en charge d'un photomultiplicateur.}
\end{figure}

\FloatBarrier

\subsection{Aquisition de donn{\'e}es}
\label{sec:materiel}
Nous allons à présent passer en revue les différents modules utilisés pour la prise de données.

\textbf{Alimentation} [figure \ref{fig:HV}] :\\
Les modules d'alimentation haute-tension permettent de contrôler la tension appliquée aux PMs et à l'OM.

\textbf{Scaler} [figure \ref{fig:scaler1}] :\\
Cette unité permet de compter les impulsions logiques entrantes. Le module de mesure standard CAMAC possède 12 entrées indépendantes. Il existe également des modules NIM possédant la même fonction. Ces derniers fonctionnent sans acquisition de données et permettent de lire directement les résultats sur un écran intégré.

\textbf{Fan-in-fan-out (FIFO)} [figure  \ref{fig:fifo}] :\\
Cette unité permet de dédoubler le signal. Il en existe 2 types:
\begin{itemize}
\item Logic FIFO : sort un signal logique
\item Linear FIFO : sort un signal analogique
\end{itemize}

\textbf{Unité de délai} [figure \ref{fig:delay}] :\\
Cette unité sort un signal identique au signal d'entrée en lui ajoutant un délai.

\textbf{Time to Digital Converter (TDC)} [figure \ref{fig:tdc}] :\\
Ce module permet de reconnaître un évènement et de fournir une représentation digital du moment auquel il est survenu. Ces unités sont communément utilisées pour mesurer un intervalle de temps et le convertir en une sortie digitale.

\textbf{Analogue to Digital Converter (ADC)} [figure \ref{fig:adc}] :\\
Ce module mesure une charge ou une différence de potentiel. 
La charge, présentée à l’entrée pendant un intervalle de temps fixé par un signal logique fourni à l’ADC (gate), est accumulée sur un condensateur. Dans un second temps, on laisse le condensateur se décharger au travers d’un circuit RC et on compte le temps nécessaire à cette décharge. Pour ce faire, on compte les impulsions d’un oscillateur, au moyen d’une échelle de comptage (scaler). Chaque module possède 12 entrées.

\textbf{Discriminateur} [figure \ref{fig:discriminator}] :\\
Donne un signal logique si le signal d'entée atteint un certain seuil. Pour une unité de logique standard NIM, le niveau "vrai" est fixé à -800 mV. Le seuil et la largeur du signal analogique peuvent être modifié.

\textbf{Unité de coïncidence} [figure \ref{fig:coincidence}] :\\
Cette unité donne un signal logique à la sortie lorsque les signaux d'entrée coïncident temporellement. On peut sélectionner la durée du signal sortant.

\textbf{Unité logique programmable} [figure  \ref{fig:programmable}]  :\\
Ce module permet de programmer les sorties en fonction des signaux d'entrée selon une table de logique. Il possède 4 entrées ainsi que 4 sorties, ce qui permet d'obtenir 16 lignes de logique. La durée de la sortie est déterminée par le temps durant lequel la coïncidence a lieu.

\begin{figure}[p]
    \centering
    \begin{subfigure}[t]{0.2\textwidth}
        \includegraphics[height=0.25\textheight, width=\textwidth, keepaspectratio]{figures/Alim1.png}
        \caption{Alimentation haute tension}
        \label{fig:HV}
    \end{subfigure}
    \hfill
    \begin{subfigure}[t]{0.2\textwidth}
        \includegraphics[height=0.25\textheight, width=\textwidth, keepaspectratio]{figures/scaler.png}
        \caption{Scaler NIM}
        \label{fig:scaler1}
    \end{subfigure}
    \hfill
    \begin{subfigure}[t]{0.2\textwidth}
        \includegraphics[height=0.25\textheight, width=\textwidth, keepaspectratio]{figures/FanInFanOut.png}
        \caption{Distributeur de signal \emph{fan-in-fan-out}}
        \label{fig:fifo}
    \end{subfigure}
    \hfill
    \begin{subfigure}[t]{0.2\textwidth}
        \includegraphics[height=0.25\textheight, width=\textwidth, keepaspectratio]{figures/delay.png}
        \caption{Retardeur de signal}
        \label{fig:delay}
    \end{subfigure}
    
	\vspace{1cm}
    \begin{subfigure}[t]{0.2\textwidth}
        \includegraphics[height=0.25\textheight, width=\textwidth, keepaspectratio]{figures/NIM_TDC.png}
        \caption{Time to Digital Converter (TDC)}
        \label{fig:tdc}
    \end{subfigure}
    \hfill	
    \begin{subfigure}[t]{0.25\textwidth}
        \includegraphics[height=0.25\textheight, width=\textwidth, keepaspectratio]{figures/gate.png}
        \caption{Analogue to Digital Converter (ADC)}
        \label{fig:adc}
    \end{subfigure}    
    \hfill
    \begin{subfigure}[t]{0.45\textwidth}
        \includegraphics[height=0.3\textheight, width=\textwidth, keepaspectratio]{figures/Discriminateur.png}
        \caption{Discriminateur}
        \label{fig:discriminator}
    \end{subfigure}

	\vspace{1cm}    
    \begin{subfigure}[t]{0.45\textwidth}
        \includegraphics[height=0.3\textheight, width=\textwidth, keepaspectratio]{figures/UniteDeCoincidence_crop.png}
        \caption{Unit{\'e} de co{\"i}ncidence}
        \label{fig:coincidence}
    \end{subfigure}
    \hfill
    \begin{subfigure}[t]{0.45\textwidth}
        \includegraphics[height=0.3\textheight, width=\textwidth, keepaspectratio]{figures/UniteLogiqueProgrammable.png}
        \caption{Unit{\'e} logique programmable}
        \label{fig:programmable}
    \end{subfigure}
    \caption{Appareils utilis{\'e}s dans les différents dispositifs.}
    \label{fig:devices}
\end{figure}
