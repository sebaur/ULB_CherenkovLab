\section{Les dispositifs experimentaux}
\subsection{Dispositif 1: L'effect Cherenkov d'\'electrons}

\subsection{Dispositif 2: L'effect Cherenkov de muons}

\subsection{Le materi\'el experimental}

\subsection{Agenda tentative}
\textbf{Lundi}
\begin{itemize}
\item Prise de connaissance avec le mat\'eriel.
\item V\'erifier signal analogique PM et sortie correspondante du discriminateur avec l'oscilloscope.
\item \'Ecrire la table de v\'erit\'e pour la logique d'acquisition ou trigger.
\item En utilisant les scaler NIM, calculer l’\'efficacit\'e de PM2 vs PM1 et PM3 (dispositif 2), ou PM1 vs PM2 et OM (dispositif 1) en fonction de la haute tension et le seui du discriminateur. Mesurer aussi les taux de PM1 (PM2) en fonction de la haute sension et le seuil.
\end{itemize}
\vspace{\baselineskip}
\textbf{Mardi}
\begin{itemize}
\item Calibrage de l'ADC: Introduire une charge connue \`a l'entr\'ee de l’ADC. Prendre des donn\'ees pour diff\'erentes charges avec le logiciel ''ADC``. D\'eterminer le param\`etre de conversion de num\'ero de coups d’ADC vers charge
\item Preparer l'acquisition de donn\'ees, verifier avec l'oscilloscope que les signaux et le gate arrivent en m\^eme temps \`a l'ADC.
\item D\'emarrer l'acquisition de donn\'ees avec LabView pour obternir le spectre en charge du PMT quand il y a de signal.
\end{itemize}
\vspace{\baselineskip}
\textbf{Mercredi}
\begin{itemize}
\item Construire la logique d'acquisition pour le bruit de fond.
\item \'Ecrire la table de v\'erit\'e pour d\'efinir un \'ev\'enement de bruit.
\item V\'erifier que le gate et le signal du PMT arrivent simultan\'ement \`a l'ADC.
\item D\'emarrer l'acquisition de donn\'ees pour obtenir le spectre en charge quand il n'y a pas de signal.
\end{itemize}
\vspace{\baselineskip}
\textbf{Jeudi et Vendredi}
\begin{itemize}
\item D\'eveloppement d'un programme de g\'en\'eration Monte Carlo d'une distribution normale et faire le fit de votre distribution Monte Carlo.
\item D\'eveloppement des programmes d'analyse.
\item Jour limit pour presenter les r\'esultats des exercices.
\item Pr\'eparation de la pre\'esentation.
\end{itemize}

\pagebreak
